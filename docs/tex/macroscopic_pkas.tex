\section{Macroscopic pKas for a single ligand species}

We would like to use the calculated effective energy differences to
determine macroscopic pKas (i.e. the equilibrium constant that
describes the populations of various macrostates). We will use proton
binding as an example but note that this derivation can apply to any
binding ligand as long as it is the sole binding species.

We define the macroscopic pKa as:

\begin{equation}
  pKa_m \equiv -\log \left(\frac{[C^-][H^+]}{[CH] c_0} \right)
\end{equation}

Where $[C^-]$ is the concentration of protonated complexes, $[H^+]$ is
the concentration of protons, $[CH]$ is the concentration of
protonated complexes, and $c_0$ is the standard state concentration of
hydrogen. Using the properties of logarithmic functions:

\begin{align*}
  pKa_m &\equiv -\log \left(\frac{[C^-][H^+]}{[CH] c_0} \right) \\
  &= -\log \left( \frac{[C^-]}{[CH]} \right) + pH \\
  &= -\log \left(\frac{P(C^-)}{P(CH)} \right) + pH \\
  &= -\ln \left( \frac{P(C^-)}{P(CH)} \right)/\ln(10) + pH \\
  &= \frac{\beta \Delta G_{C^-,CH}}{\ln 10} + pH
\end{align*}

where $\Delta G_{C^-,CH}$ is the effective free energy difference of
protonation, which is: 

\begin{equation}
\Delta G_{N-1,N} = \beta^{-1} \ln \left[ \frac{\sum_s \exp(-\beta
    \Delta G_s)\delta_{N_s,N-1}}{\sum_s \exp(-\beta \Delta
    G_s)\delta_{N_s,N}} \right]
\label{macrostateenergydiff}
\end{equation}

where the $\delta_{N_s,N}$ and $\delta_{N_s,N-1}$ will pick out states
that contain the correct number of protonated residues. For sake of
clarity, we select as a reference energy, the energy any deprotonated
microstate. The motivation for this choice becomes clear when
considering the free energy of (microscopic)
protonation/deprotonation. To protonated/deprotonate a residue, the free energy differnce is:

\begin{equation}
  \Delta G_{prot} = -\Delta G_{dep} = \beta^{-1} \ln(10)(pH-pKa)
\end{equation}

Consider a system of three protonatable residues. We will represent
the protonation state by a string of 1s and 0s. For example, 111
represents that state where all residues are protonated (N=3) while
000 represents the state where no residues are protonated (N=0). These
two examples, of course are macrostates with no degenerate
microstates. We have two other macrostates of interest, N=1 and
N=2. The N=1 macrostate has a three fold degeneracy (001,010,100) and
the N=2 macrostate has a three fold degeneracy (011,101,110). In order
to go from 001 to 010, we would be required to deprotonate residue
three and then protonate residue two. The energy difference between
these two microstates is then:

\begin{equation}
  \Delta G_{001,010} = \beta^{-1} \ln(10)\left[ (-pH + pKa_{000,001}) + (pH - pKa_{000,010}) \right] = \beta^{-1} \ln(10) \left[ pKa_{000,001} - pKa_{000,010}\right]
\end{equation}

A similar argument can be made for free energy difference between 001
and 100. Because of this behavior, we see that the energy difference
between two microstates is only dependent on the temperature and the
difference of the pKas of each state. In the case of moving between
microstates from different macrostates, there is a first order pH
dependence, as well as a linear combination of pKas that is a function
of the connectivity of the graph. We can select any reference state we
would like. In our case, we select any $\Delta G$ from the numerator
of the argument of the natural logarithm. By doing this, all terms in
the numerator have no pH dependence and are all just functions of the
connectivity, which we will call $\eta_i$.

We then find that (\ref{macrostateenergydiff}) can be written as:

\begin{align*}
  \Delta G_{N-1,N} &= \beta^{-1} \ln \left[ \frac{\sum_s \exp(-\eta_s) \exp(-\ln(10)pH) \delta_{N_s,N-1}}{\sum_s \exp(-\eta_s) \delta_{N_s,N}} \right] \\
  &= \beta^{-1} \ln \left[ \frac{\exp( - \ln(10)pH) \sum_s \exp(-\eta_s) \delta_{N_s,N-1}}{\sum_s \exp(-\eta_s) \delta_{N_s,N}} \right] \\
  &= - \beta^{-1} \ln(10) pH + \beta^{-1} \ln \left[ \frac{\sum_s \exp(-\eta_s) \delta_{N_s,N-1}}{\sum_s \exp(- \eta_s) \delta_{N_s,N}} \right] 
\end{align*}

Plugging this into our expression for the $pKa_m$, we find that the pH
dependence drops out and our result is:

\begin{align*}
  pKa_m = \ln \left[ \frac{\sum_s \exp(-\eta_s) \delta_{N_s,N-1}}{\sum_s \exp(- \eta_s) \delta_{N_s,N}} \right] / \ln(10)
\end{align*}

We conclude that the macroscopic pKas are independent of pH. More
generally, we have shown that the macroscopic equilibrium constants
are independent of the ligand concentration.
